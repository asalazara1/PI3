\documentclass[12pt,letterpaper]{article}
\usepackage{geometry}
\usepackage[utf8]{inputenc}
\usepackage{natbib}
\usepackage{lscape}
\usepackage{amsmath}
\usepackage{amsfonts}
\usepackage{amssymb}
\usepackage{graphicx}
\usepackage{subfig}
\usepackage{float}
\usepackage{algpseudocode}
\usepackage{fancyhdr}
\usepackage{comment}
\usepackage{hyperref}

\usepackage[english]{babel}
\geometry{verbose,letterpaper,tmargin=3cm,bmargin=3cm,lmargin=2.5cm,rmargin=2.5cm}

% Tables
\usepackage{multirow}
\usepackage[table]{xcolor}

% Avoid to split words at the end of each line
\usepackage{hyphenat}

\usepackage{setspace}

\usepackage{natbib} 
\setcitestyle{numbers,square}

\usepackage{notoccite}


\singlespacing

\begin{document}

% BEGIN TITLE PAGE
\begin{titlepage}

\Large
\sffamily

\begin{center}
  \begin{tabular}{c}
    \includegraphics[width=0.30\textwidth]{logo-eafit.png}
  \end{tabular}
\end{center}

\vfill
\begin{center}
  \LARGE Deep learning techniques for Blood Flow simulation based on Navier-Stokes Equations
\end{center}

\vspace*{1cm}
\centerline{\LARGE Alejandro Salazar Arango\footnotemark}  
\footnotetext{Mathematical Engineering Student. Universidad EAFIT asalazara1@eafit.edu.co}
\vfill

\begin{center}
Advisor(s): \\
Cristhian David Montoya Zambrano\footnote{School of Applied Sciences and Engineering. Universidad EAFIT cdmontoyaz@eafit.edu.co}   \\
\end{center}

\vfill

\begin{center}
  \large
    Research practice 3 \\
  Research proposal \\
  Mathematical Engineering\\
  School of Applied Sciences and Engineering\\
  Universidad EAFIT \\
\end{center}

\vfill
\centerline{August 2023}
\vspace*{0.7cm}
\end{titlepage}

% END TITLE PAGE

\begin{abstract}
    
  The purpose of this paper is to 
  
  present a simulation of blood flow through the carotid artery in two dimensions. The simulation is performed on a section of the artery extracted from an angiography, on which modifications are made to mimic the presence of stenosis in the artery. Finite element method is used to solve the linear simplification of the Navier-Stokes equations, seeking to identify variations in the pressure and velocity fields of the fluid and to analyze the impact of mild and severe cases of stenosis caused by atherosclerosis on human hemodynamics.\\
  
    \noindent \textbf{Keywords:} Finite element method, Computational fluid dynamics, Hemodynamics, Ischemia, Stroke 
    \end{abstract}

\section{Introduction}

\begin{comment}

  (\textcolor{red}{4 a 5 párrafos})

Debe incluir:

\begin{itemize}
\item Contexto corto para ubicar al lector.
\item Corta descripción del problema u oportunidad de investigación.
\item Aporte de la investigación.
\item Párrafo que justifique la importancia de la investigación.
\end{itemize}

\end{comment}


Partial differential equations (PDEs) are one of the most powerful tools for the study of physical and biological phenomena. They allow us to represent in a simplified way, the dynamics between the spatial and temporal variables of any phenomenon and to take these dynamics to a mathematical model, whose solution describes in a great way the behavior of the studied system\cite{logan2014applied}. This has led partial differential equations to become a tool of great use in areas such as fluid dynamics, electromagnetism, optics, magnetism, among others\cite{farlow1993partial}; these areas being dominated by PDEs such as Maxwell's Equations and Navier-Stokes Equations. \\

However, PDEs, given their complex structure, are usually models whose analytical solutions are difficult or impossible to find, being even the solutions already found for some PDEs so complex that it is preferred to use other methods to solve the problem\cite{strauss2007partial}. This is why, in recent years, the development of multiple numerical methods to solve this type of equations has been one of the major topics of study of the scientific community. In addition, numerical methods for solving PDEs are a really important topic of study given that although there are already multiple methods developed, if these are not applied correctly can lead to solutions far from the real solution of the problem and therefore lead to erroneous conclusions about the behavior of the system under study.\\

One of the most commonly used numerical method for solving Partial Differential Equations is the Finite Element Method (FEM), a numerical method consisting on the discretization of the problem's domain, to later solve a variational formulation of the problem with determined test functions, to obtain an algebraic system of equations, which, when solved, gives the approximated solution of the problem. However, due to the substantial number of degrees of freedom it needs in order to correctly solve the PDE, it remains as an extremely expensive method both in terms of CPU and memory demand, therefore not being too useful for real-time contexts\cite{hesthaven2018non} \cite{PINNQuarteroni}. Consequently, models such as physically informed neural networks (PINNs) have been developed and have become important methods in the study of PDEs.\\

PINNs are quite useful in this type of problems since, by incorporating the physical information of the system under study in the training of the neural networks, allowing the imposition of boundary conditions and governing equations directly on the network architecture, it facilitates the adaptation of the network to different conditions and geometries on which to solve the problem, in addition to potentially even improving the accuracy of the numerical solutions.

Therefore thi project focuses on the implementation of multiple Physical Informed Neural Networks (PINNs) architechtures for solving Fluid Dynamic problems, seeking to observe and compare the performance of these architectures, aiming for a deeper understanding of the advantages and limitations of PINNs in the context of Fluid dynamics simulations.


\begin{comment}
  Parrafo comentando los principales métodos implementados en CFD (principamente FEM) comentar el alto número de grados de libertad e introducir RBM y PINN utilizando NN.
  
  Parrafo comentando rápidamente el objetivo de la investigación, como se realizará y el aporte esperado.
\end{comment}


\section{State of the art}

\begin{comment}
(\textcolor{red}{5 a 6 párrafos})
Describe las principales referencias relacionadas con el problema. Este estado
del arte puede referirse al problema o aplicación específica, o bien a los
métodos aplicados para solucionarlo. No debes olvidar incluir los trabajos más
importantes y los más recientes.

\end{comment}

Given its important on a great amount of biological, medical, and physical fields, PDEs are one of the most studied topics in applied mathematics, and therefore, a large body of literature on the subject can be found. The following is a review of some of the literature consulted on PINNs and RBMs and their applications in EDPs, where a collection of both theoretical and applied papers are presented.\\

On the other hand, although PINNs are new techniques in the solution of PDEs, they have also been shown to be a topic of great interest to the scientific community and therefore it is not difficult to find literature on the subject, not only for forward problems but for inverse problems as well. On this matter, Berg \& Nyström \cite{berg2021neural} , Meng \& Karniadakis \cite{meng2020composite} and Raissi, Perdikaris \& Karniadakis \cite{raissi2019physics} present three works of interest in which they use PINNs as a solution method for inverse problems with multiple examples and making a focus on the explanation of how PINNs work on every presented examples, as well as a mathematical background on the method itself.\\

Some works about the usage of PINN for fluid dynamics problems can be found as well. A first interesting work on this topic can be found in the paper from Moschou \textit{et al.} \cite{moschou2023physics} who present an application of PINNs on the modeling of astrophysical shocks, focusing on the solar termination shock, a phenomenon described using the fluid dynamics conservation laws. In agreement Ciao \textit{et al.} \cite{xiao2023physics} developed an application of PINN for the turbulent mixing induced by the Rayleigh-Taylor instability, using the Reynolds-Averaged Navier–Stokes Equations for modeling the problem. And finally, Lino \textit{et al.} \cite{lino2023current} exhibit a review of the current and emerging deep-learning methods used on fluid dynamics simulation.

\section{Proposed methodology}

\begin{comment}

 (\textcolor{red}{5 o 6 párrafos})
Describe en este apartado los métodos, técnicas, algoritmos, etc. que se
utilizarán durante la ejecución del proyecto.

Incluye aspectos como:

\begin{itemize}
\item ¿Qué métodos se suelen usar para responder la pregunta de investigación?
\item ¿Por qué seleccionaste el método que usarás?
\item Describe el método: Supuestos básicos, ventajas y desventajas del método.
\end{itemize}
\end{comment}

As a means to achieve the objectives set for this project, a 3-stage methodology was proposed in which various mathematical methods were used. In the first stage a version of the Broyden–Fletcher–Goldfarb–Shanno algorithm, the L-BFGS-B optimization method, was implemented, which is the optimization method used for the training of the PINNs implemented. The second stage consisted on the implementation, training and validation of multiple PINN achitectures, which were used for solving both Burgers and Navier Stokes equations with a variety of cases for comparing the capacity of these Neural Networks for solving these problems. Finally the third stage was the aalysis of results and the comparison of the performances of the models implemented on the previous stage.\\

Below both L-BFGS-B an PINN are explained with more detail for the reader to understand the nature of both algorithms and how they work.

\subsection{The Broyden–Fletcher–Goldfarb–Shanno algorithm}

The Broyden-Fletcher-Goldfarb-Shanno (BFGS) algorithm is an iterative method for numerical optimization, specifically used for finding the minimum of a function. It falls under the category of quasi-Newton methods and is particularly well-suited for problems where the objective function is smooth and has continuous derivatives. The main goal of the BFGS algorithm is to iteratively update an approximation of the inverse Hessian matrix, which represents the local curvature of the objective function, to efficiently converge towards the minimum.\\

At each iteration, the BFGS algorithm computes the gradient of the objective function with respect to the parameters, in order to subsequently update the current estimate of the inverse Hessian matrix based on the difference in gradients and parameter updates between iterations as shown in Equation \ref{Eq:Hess}. By utilizing this information, the algorithm generates a sequence of parameter values that progressively move closer to the optimal solution. Unlike Newton's method, BFGS avoids explicitly computing the Hessian matrix, which can be computationally expensive for large problems, by iteratively updating an approximation of it \cite{BFGS}.

\begin{align}
  \label{Eq:Hess}
  \begin{split}
    s_k & =  x_{k+1} + x_{k}\\
    y_k & = \nabla f(x_{k+1}) - \nabla f(x_{k})\\
    H_{k+1}^{-1} & = (I-\frac{sy^T}{y^Ts})H_{k}^{-1}\frac{ys^T}{y^Ts} + \frac{ss^T}{y^Ts} 
  \end{split}
\end{align}\\
BFGS' use of an approximation of the inverse Hessian matrix allows for a balance between efficiency and accuracy, making it a popular choice for optimizing smooth and high-dimensional functions. However, BFGS does not guarantee convergence to a global minimum and may converge to a local minimum based on the starting point and the characteristics of the objective function. 

\subsection{Physical Informed neural Networks}

Physics-informed neural networks (PINNs) have emerged as a promising approach to solve PDEs with remarkable efficiency and accuracy. These combine the power of neural networks with the governing equations of physical systems, offering a seamless integration of data-driven learning and domain-specific physics. They work by incorporating the residual error of the governing equations into the neural network training process, gaining a unique ability to capture the underlying physics of the model while learning from sparse and noisy data\cite{raissi2019physics}. \\

Unlike traditional methods, PINNs avoid the need for explicit mesh discretization and provide continuous solutions throughout the domain, dynamically adapting to complex geometries. Their versatility, combined with the ability to leverage existing deep learning frameworks, positions PINNs as a transformative tool for advancing EDP-based modeling and simulation in interdisciplinary domains.\\

\section{Results}

For this project three set of equations were considered with multiple scenarios and architechtures each. The main results obtained for each set of equations is presented below

\subsection{1-dimentional Burgers' Equation}

For the 1-dimentional Burgers0 Equation, three cases were implemeneted each with the listed three architectures:

\begin{itemize}
  \item Architechture 1: 5 hidden layers with 20 neurons each
  \item Architechture 2: 5 hidden layers with 50 neurons each
  \item Architechture 3: 8 hidden layers with 20 neurons each
\end{itemize}

\subsubsection*{Case 1:}

For the first case the following exact solution $u(x,t)$ was considered

$$u(x,t) = e^{-t}\sin(\pi x)$$\\

which, for $\nu=\frac{1}{\pi^2}$, gives the PDE below

\begin{equation}
  \label{B1}
  u_t + u u_x - \frac{1}{\pi^2}u_{xx} = \pi e^{-2t}\sin(\pi x)\cos(\pi x) 
\end{equation}

When solving Equation \ref{B1} with the architectures explained above, the following results were obtained

\begin{table}[H]
  \begin{center}
  \begin{tabular}{ r | c  c }
  \textbf{Time} & \textbf{MSE} & \textbf{$l^2$ error} \\ \hline
  0.0 & $ 4.1341\times 10^{-8}$ & $ 2.0332\times 10^{-3} $ \\
  0.1 & $ 1.7436\times 10^{-8}$ & $ 1.3205\times 10^{-3} $ \\
  0.2 & $ 9.8793\times 10^{-9}$ & $ 9.9395\times 10^{-4} $ \\
  0.3 & $ 7.5820\times 10^{-9}$ & $ 8.7074\times 10^{-4} $ \\
  0.4 & $ 5.1086\times 10^{-9}$ & $ 7.1475\times 10^{-4} $ \\
  0.5 & $ 2.1645\times 10^{-9}$ & $ 4.6525\times 10^{-4} $ \\
  0.6 & $ 1.1935\times 10^{-9}$ & $ 3.4546\times 10^{-4} $ \\
  0.7 & $ 2.6267\times 10^{-9}$ & $ 5.1251\times 10^{-4} $ \\
  0.8 & $ 2.9190\times 10^{-9}$ & $ 5.4027\times 10^{-4} $ \\
  0.9 & $ 8.5705\times 10^{-10}$ & $ 2.9275\times 10^{-4} $ \\
  1.0 & $ 1.0075\times 10^{-8}$ & $ 1.0037\times 10^{-3} $ \\
  \end{tabular}
  \caption{Results for the first architecture in the first case of Burgers Equations}
  \label{tab:B11}
  \end{center}
  \end{table}

  \begin{table}[H]
    \begin{center}
    \begin{tabular}{ r | c  c }
    \textbf{Time} & \textbf{MSE} & \textbf{$l^2$ error} \\ \hline
    0.0 & $ 4.8725\times 10^{-9}$ & $ 6.9803\times 10^{-4} $ \\
    0.1 & $ 8.4316\times 10^{-9}$ & $ 9.1824\times 10^{-4} $ \\
    0.2 & $ 1.3853\times 10^{-8}$ & $ 1.1770\times 10^{-3} $ \\
    0.3 & $ 1.4626\times 10^{-8}$ & $ 1.2094\times 10^{-3} $ \\
    0.4 & $ 1.0354\times 10^{-8}$ & $ 1.0176\times 10^{-3} $ \\
    0.5 & $ 5.4203\times 10^{-9}$ & $ 7.3623\times 10^{-4} $ \\
    0.6 & $ 4.0029\times 10^{-9}$ & $ 6.3269\times 10^{-4} $ \\
    0.7 & $ 5.2350\times 10^{-9}$ & $ 7.2353\times 10^{-4} $ \\
    0.8 & $ 4.1608\times 10^{-9}$ & $ 6.4504\times 10^{-4} $ \\
    0.9 & $ 1.2808\times 10^{-9}$ & $ 3.5789\times 10^{-4} $ \\
    1.0 & $ 2.4014\times 10^{-8}$ & $ 1.5497\times 10^{-3} $ \\
    \end{tabular}
    \caption{Results for the second architecture in the first case of Burgers Equations}
    \label{tab:B12}
    \end{center}

    \end{table}
    \begin{table}[H]
      \begin{center}
      \begin{tabular}{ r | c  c }
      \textbf{Time} & \textbf{MSE} & \textbf{$l^2$ error} \\ \hline
      0.0 & $ 2.7134\times 10^{-9}$ & $ 5.2090\times 10^{-4} $ \\
      0.1 & $ 1.2367\times 10^{-9}$ & $ 3.5167\times 10^{-4} $ \\
      0.2 & $ 4.1095\times 10^{-9}$ & $ 6.4105\times 10^{-4} $ \\
      0.3 & $ 6.9692\times 10^{-9}$ & $ 8.3482\times 10^{-4} $ \\
      0.4 & $ 6.4658\times 10^{-9}$ & $ 8.0410\times 10^{-4} $ \\
      0.5 & $ 4.1114\times 10^{-9}$ & $ 6.4120\times 10^{-4} $ \\
      0.6 & $ 2.0376\times 10^{-9}$ & $ 4.5139\times 10^{-4} $ \\
      0.7 & $ 8.9270\times 10^{-10}$ & $ 2.9878\times 10^{-4} $ \\
      0.8 & $ 4.2240\times 10^{-10}$ & $ 2.0552\times 10^{-4} $ \\
      0.9 & $ 3.2816\times 10^{-10}$ & $ 1.8115\times 10^{-4} $ \\
      1.0 & $ 1.0638\times 10^{-9}$ & $ 3.2616\times 10^{-4} $ \\
      \end{tabular}
      \caption{Results for the third architecture in the first case of Burgers Equations}
      \label{tab:B13}
      \end{center}
      \end{table}  
\subsubsection*{Case 2:}

For the second case the following exact solution $u(x,t)$ was considered

$$u(x,t) = e^{t}(x-x^2)$$\\

which, for $\nu=1$, gives the PDE below

\begin{equation}
  \label{B2}
  u_t + u u_x - u_{xx} = e^t (x-x^2) + e^{2t} (x+x^2) (1-2x) + 2e^t
\end{equation}

When solving Equation \ref{B2} with the architectures explained above, the following results were obtained

\begin{table}[H]
  \begin{center}
  \begin{tabular}{ r | c  c }
  \textbf{Time} & \textbf{MSE} & \textbf{$l^2$ error} \\ \hline
  0.0 & $ 2.7535\times 10^{-1}$ & $ 5.2474 $ \\
  0.1 & $ 1.9203\times 10^{-1}$ & $ 4.3821 $ \\
  0.2 & $ 1.2686\times 10^{-1}$ & $ 3.5618 $ \\
  0.3 & $ 7.7156\times 10^{-2}$ & $ 2.7777 $ \\
  0.4 & $ 4.0875\times 10^{-2}$ & $ 2.0218 $ \\
  0.5 & $ 1.6563\times 10^{-2}$ & $ 1.2870 $ \\
  0.6 & $ 3.2628\times 10^{-3}$ & $ 0.5712 $ \\
  0.7 & $ 4.6056\times 10^{-4}$ & $ 0.2146 $ \\
  0.8 & $ 8.0537\times 10^{-3}$ & $ 0.8974 $ \\
  0.9 & $ 2.6332\times 10^{-2}$ & $ 1.6227 $ \\
  1.0 & $ 5.5973\times 10^{-2}$ & $ 2.3659 $ \\
  \end{tabular}
  \caption{Results for the first architecture in the second case of Burgers Equations}
  \label{tab:B21}
  \end{center}
  \end{table}

  \begin{table}[H]
    \begin{center}
    \begin{tabular}{ r | c  c }
    \textbf{Time} & \textbf{MSE} & \textbf{$l^2$ error} \\ \hline
    0.0 & $ 2.7545\times 10^{-1}$ & $ 5.2483 $ \\
    0.1 & $ 1.9191\times 10^{-1}$ & $ 4.3807 $ \\
    0.2 & $ 1.2675\times 10^{-1}$ & $ 3.5602 $ \\
    0.3 & $ 7.7123\times 10^{-2}$ & $ 2.7771 $ \\
    0.4 & $ 4.0908\times 10^{-2}$ & $ 2.0226 $ \\
    0.5 & $ 1.6612\times 10^{-2}$ & $ 1.2889 $ \\
    0.6 & $ 3.2859\times 10^{-3}$ & $ 0.5732 $ \\
    0.7 & $ 4.5513\times 10^{-4}$ & $ 0.2133 $ \\
    0.8 & $ 8.0546\times 10^{-3}$ & $ 0.8975 $ \\
    0.9 & $ 2.6372\times 10^{-2}$ & $ 1.6239 $ \\
    1.0 & $ 5.5989\times 10^{-2}$ & $ 2.3662 $ \\
    \end{tabular}
    \caption{Results for the second architecture in the second case of Burgers Equations}
    \label{tab:B22}
    \end{center}

    \end{table}
    \begin{table}[H]
      \begin{center}
      \begin{tabular}{ r | c  c }
      \textbf{Time} & \textbf{MSE} & \textbf{$l^2$ error} \\ \hline
      0.0 & $ 2.7520\times 10^{-1}$ & $ 5.2459 $ \\
      0.1 & $ 1.9223\times 10^{-1}$ & $ 4.3844 $ \\
      0.2 & $ 1.2703\times 10^{-1}$ & $ 3.5641 $ \\
      0.3 & $ 7.7172\times 10^{-2}$ & $ 2.7780 $ \\
      0.4 & $ 4.0783\times 10^{-2}$ & $ 2.0195 $ \\
      0.5 & $ 1.6462\times 10^{-2}$ & $ 1.2830 $ \\
      0.6 & $ 3.2196\times 10^{-3}$ & $ 0.5674 $ \\
      0.7 & $ 4.6870\times 10^{-4}$ & $ 0.2165 $ \\
      0.8 & $ 8.0401\times 10^{-3}$ & $ 0.8967 $ \\
      0.9 & $ 2.6244\times 10^{-2}$ & $ 1.6200 $ \\
      1.0 & $ 5.5965\times 10^{-2}$ & $ 2.3657 $ \\
      \end{tabular}
      \caption{Results for the third architecture in the second case of Burgers Equations}
      \label{tab:B23}
      \end{center}
      \end{table}  

  \subsubsection*{Case 3:}

  For the third case the following exact solution $u(x,t)$ was considered
  
  $$u(x,t) = t e^{x-x^2}$$\\
  
  which, for $\nu=1$, gives the PDE below
  
  \begin{equation}
    \label{B3}
    u_t + u u_x - u_{xx} = e^{x-x^2} + t^2 e^{2(x-x^2)}(1-2x) - t e^{x-x^2}(4x^2 - 4x - 1)
  \end{equation}
  
  When solving Equation \ref{B3} with the architectures explained above, the following results were obtained
  
  \begin{table}[H]
    \begin{center}
    \begin{tabular}{ r | c  c }
    \textbf{Time} & \textbf{MSE} & \textbf{$l^2$ error} \\ \hline

    \end{tabular}
    \caption{Results for the first architecture in the third case of Burgers Equations}
    \label{tab:B31}
    \end{center}
    \end{table}
  
    \begin{table}[H]
      \begin{center}
      \begin{tabular}{ r | c  c }
      \textbf{Time} & \textbf{MSE} & \textbf{$l^2$ error} \\ \hline

      \end{tabular}
      \caption{Results for the second architecture in the third case of Burgers Equations}
      \label{tab:B32}
      \end{center}
  
      \end{table}
      \begin{table}[H]
        \begin{center}
        \begin{tabular}{ r | c  c }
        \textbf{Time} & \textbf{MSE} & \textbf{$l^2$ error} \\ \hline
        0.0 & $ 3.9095\times 10^{-9}$ & $ 6.2526\times 10^{-4} $ \\
        0.1 & $ 4.0976\times 10^{-9}$ & $ 6.4012\times 10^{-4} $ \\
        0.2 & $ 1.2145\times 10^{-8}$ & $ 1.1020\times 10^{-3} $ \\
        0.3 & $ 1.9778\times 10^{-8}$ & $ 1.4063\times 10^{-3} $ \\
        0.4 & $ 2.1008\times 10^{-8}$ & $ 1.4494\times 10^{-3} $ \\
        0.5 & $ 1.3044\times 10^{-8}$ & $ 1.1421\times 10^{-3} $ \\
        0.6 & $ 3.4472\times 10^{-9}$ & $ 5.8713\times 10^{-4} $ \\
        0.7 & $ 1.7965\times 10^{-9}$ & $ 4.2385\times 10^{-4} $ \\
        0.8 & $ 6.2565\times 10^{-9}$ & $ 7.9098\times 10^{-4} $ \\
        0.9 & $ 4.9122\times 10^{-9}$ & $ 7.0087\times 10^{-4} $ \\
        1.0 & $ 1.9582\times 10^{-8}$ & $ 1.3994\times 10^{-3} $ \\
        \end{tabular}
        \caption{Results for the third architecture in the third case of Burgers Equations}
        \label{tab:B33}
        \end{center}
        \end{table}  

\subsection{1-dimentional Navier-Stokes Equations}

For the 1-dimentional Navier-Stokes Equations, two cases were implemeneted each with the listed four architectures:

\begin{itemize}
  \item Architechture 1: 5 hidden layers with 20 neurons each
  \item Architechture 2: 5 hidden layers with 50 neurons each
  \item Architechture 3: 8 hidden layers with 20 neurons each
  \item Architechture 4: 8 hidden layers with 50 neurons each
\end{itemize}

\subsubsection*{Case 1:}

For the first case the following exact solution $u(x,t)$ and $p(x,t)$ was considered

$$u(x,t) = e^{-t}\sin(\pi x)$$
$$p(x,t) = -\sin(\pi x)$$\\

which, for $\nu=\frac{1}{\pi^2}$, gives the PDE below

\begin{equation}
  \label{NS11}
  u_t + u u_x - \frac{1}{\pi^2}u_{xx} + p_x = \pi e^{-2t}\left(\sin(\pi x)\cos(\pi x) - \cos(\pi x)\right) 
\end{equation}

When solving Equation \ref{NS11} with the architectures explained above, the following results were obtained

\begin{table}[H]
  \begin{center}
  \begin{tabular}{ r | c  c | c  c}
    & \multicolumn{2}{c |}{$u(x,t)$} & \multicolumn{2}{c}{$p(x,t)$} \\ \hline
  \textbf{Time} & \textbf{MSE} & \textbf{$l^2$ error}  & \textbf{MSE} & \textbf{$l^2$ error} \\ \hline
  0.0 & $ 1.9646\times 10^{-7}$ & $ 0.0044 $ & $ 9.8202\times 10^{-6 }$ & $ 0.0313$ \\
  0.1 & $ 6.8350\times 10^{-7}$ & $ 0.0083 $ & $ 1.2453\times 10^{-5 }$ & $ 0.0353$ \\
  0.2 & $ 1.7235\times 10^{-6}$ & $ 0.0131 $ & $ 1.6410\times 10^{-5 }$ & $ 0.0405$ \\
  0.3 & $ 3.2262\times 10^{-6}$ & $ 0.0180 $ & $ 2.1246\times 10^{-5 }$ & $ 0.0461$ \\
  0.4 & $ 5.4656\times 10^{-6}$ & $ 0.0234 $ & $ 2.7291\times 10^{-5 }$ & $ 0.0522$ \\
  0.5 & $ 8.6866\times 10^{-6}$ & $ 0.0295 $ & $ 3.4859\times 10^{-5 }$ & $ 0.0590$ \\
  0.6 & $ 1.3048\times 10^{-5}$ & $ 0.0361 $ & $ 4.3914\times 10^{-5 }$ & $ 0.0663$ \\
  0.7 & $ 1.8639\times 10^{-5}$ & $ 0.0432 $ & $ 5.3911\times 10^{-5 }$ & $ 0.0734$ \\
  0.8 & $ 2.5493\times 10^{-5}$ & $ 0.0505 $ & $ 6.3784\times 10^{-5 }$ & $ 0.0799$ \\
  0.9 & $ 3.3628\times 10^{-5}$ & $ 0.0580 $ & $ 7.2126\times 10^{-5 }$ & $ 0.0849$ \\
  1.0 & $ 4.3075\times 10^{-5}$ & $ 0.0656 $ & $ 7.7515\times 10^{-5 }$ & $ 0.0880$ \\
  \end{tabular}
  \caption{Results for the first architecture in the first case of the 1-dimensional Navier-Stokes Equations}
  \label{tab:NS111}
  \end{center}
  \end{table}

  \begin{table}[H]
    \begin{center}
    \begin{tabular}{ r | c  c | c  c}
      & \multicolumn{2}{c |}{$u(x,t)$} & \multicolumn{2}{c}{$p(x,t)$} \\ \hline
    \textbf{Time} & \textbf{MSE} & \textbf{$l^2$ error}  & \textbf{MSE} & \textbf{$l^2$ error} \\ \hline
    0.0 & $ 9.8856\times 10^{-9}$ & $ 0.0010 $ & $ 2.4502\times 10^{-6 }$ & $ 0.0157$ \\
    0.1 & $ 2.4990\times 10^{-7}$ & $ 0.0050 $ & $ 2.4867\times 10^{-6 }$ & $ 0.0158$ \\
    0.2 & $ 7.4739\times 10^{-7}$ & $ 0.0086 $ & $ 3.3587\times 10^{-6 }$ & $ 0.0183$ \\
    0.3 & $ 1.4044\times 10^{-6}$ & $ 0.0119 $ & $ 5.0945\times 10^{-6 }$ & $ 0.0226$ \\
    0.4 & $ 2.2494\times 10^{-6}$ & $ 0.0150 $ & $ 7.7558\times 10^{-6 }$ & $ 0.0278$ \\
    0.5 & $ 3.3342\times 10^{-6}$ & $ 0.0183 $ & $ 1.1329\times 10^{-5 }$ & $ 0.0337$ \\
    0.6 & $ 4.7053\times 10^{-6}$ & $ 0.0217 $ & $ 1.5696\times 10^{-5 }$ & $ 0.0396$ \\
    0.7 & $ 6.3922\times 10^{-6}$ & $ 0.0253 $ & $ 2.0645\times 10^{-5 }$ & $ 0.0454$ \\
    0.8 & $ 8.3984\times 10^{-6}$ & $ 0.0290 $ & $ 2.5889\times 10^{-5 }$ & $ 0.0509$ \\
    0.9 & $ 1.0697\times 10^{-5}$ & $ 0.0327 $ & $ 3.1097\times 10^{-5} $ & $ 0.0558$ \\
    1.0 & $ 1.3233\times 10^{-5}$ & $ 0.0364 $ & $ 3.5934\times 10^{-5 }$ & $ 0.0599$ \\
    \end{tabular}
    \caption{Results for the second architecture in the first case of the 1-dimensional Navier-Stokes Equations}
    \label{tab:NS112}
    \end{center}
    \end{table}
  
    \begin{table}[H]
      \begin{center}
      \begin{tabular}{ r | c  c | c  c}
        & \multicolumn{2}{c |}{$u(x,t)$} & \multicolumn{2}{c}{$p(x,t)$} \\ \hline
      \textbf{Time} & \textbf{MSE} & \textbf{$l^2$ error}  & \textbf{MSE} & \textbf{$l^2$ error} \\ \hline
      0.0 & $ 6.1127\times 10^{-7}$ & $ 0.0078 $ & $ 1.1934\times 10^{-4} $ & $ 0.1092$ \\
      0.1 & $ 1.0509\times 10^{-5}$ & $ 0.0324 $ & $ 2.1964\times 10^{-4} $ & $ 0.1482$ \\
      0.2 & $ 3.9397\times 10^{-5}$ & $ 0.0628 $ & $ 3.3119\times 10^{-4} $ & $ 0.1820$ \\
      0.3 & $ 8.3767\times 10^{-5}$ & $ 0.0915 $ & $ 4.4435\times 10^{-4} $ & $ 0.2108$ \\
      0.4 & $ 1.3836\times 10^{-4}$ & $ 0.1176 $ & $ 5.5427\times 10^{-4} $ & $ 0.2354$ \\
      0.5 & $ 1.9779\times 10^{-4}$ & $ 0.1406 $ & $ 6.5851\times 10^{-4} $ & $ 0.2566$ \\
      0.6 & $ 2.5804\times 10^{-4}$ & $ 0.1606 $ & $ 7.5561\times 10^{-4} $ & $ 0.2749$ \\
      0.7 & $ 3.1720\times 10^{-4}$ & $ 0.1781 $ & $ 8.4431\times 10^{-4} $ & $ 0.2906$ \\
      0.8 & $ 3.7555\times 10^{-4}$ & $ 0.1938 $ & $ 9.2328\times 10^{-4} $ & $ 0.3039$ \\
      0.9 & $ 4.3517\times 10^{-4}$ & $ 0.2086 $ & $ 9.9122\times 10^{-4} $ & $ 0.3148$ \\
      1.0 & $ 4.9946\times 10^{-4}$ & $ 0.2235 $ & $ 1.0470\times 10^{-3} $ & $ 0.3236$ \\
      \end{tabular}
      \caption{Results for the third architecture in the first case of the 1-dimensional Navier-Stokes Equations}
      \label{tab:NS113}
      \end{center}
      \end{table}

      \begin{table}[H]
        \begin{center}
        \begin{tabular}{ r | c  c | c  c}
          & \multicolumn{2}{c |}{$u(x,t)$} & \multicolumn{2}{c}{$p(x,t)$} \\ \hline
        \textbf{Time} & \textbf{MSE} & \textbf{$l^2$ error}  & \textbf{MSE} & \textbf{$l^2$ error} \\ \hline
        0.0 & $ 1.1785\times 10^{-7}$ & $ 0.0034 $ & $ 1.1033\times 10^{-5} $ & $ 0.0332$ \\
        0.1 & $ 3.4148\times 10^{-7}$ & $ 0.0058 $ & $ 7.1700\times 10^{-6} $ & $ 0.0268$ \\
        0.2 & $ 9.7519\times 10^{-7}$ & $ 0.0099 $ & $ 5.3050\times 10^{-6} $ & $ 0.0230$ \\
        0.3 & $ 2.1471\times 10^{-6}$ & $ 0.0147 $ & $ 5.0460\times 10^{-6} $ & $ 0.0225$ \\
        0.4 & $ 4.0106\times 10^{-6}$ & $ 0.0200 $ & $ 6.1678\times 10^{-6} $ & $ 0.0248$ \\
        0.5 & $ 6.6570\times 10^{-6}$ & $ 0.0258 $ & $ 8.5383\times 10^{-6} $ & $ 0.0292$ \\
        0.6 & $ 1.0104\times 10^{-5}$ & $ 0.0318 $ & $ 1.2097\times 10^{-5} $ & $ 0.0348$ \\
        0.7 & $ 1.4311\times 10^{-5}$ & $ 0.0378 $ & $ 1.6868\times 10^{-5} $ & $ 0.0411$ \\
        0.8 & $ 1.9196\times 10^{-5}$ & $ 0.0438 $ & $ 2.2987\times 10^{-5} $ & $ 0.0479$ \\
        0.9 & $ 2.4648\times 10^{-5}$ & $ 0.0496 $ & $ 3.0768\times 10^{-5} $ & $ 0.0555$ \\
        1.0 & $ 3.0525\times 10^{-5}$ & $ 0.0552 $ & $ 4.0788\times 10^{-5} $ & $ 0.0639$ \\
        \end{tabular}
        \caption{Results for the fourth architecture in the first case of the 1-dimensional Navier-Stokes Equations}
        \label{tab:NS114}
        \end{center}
        \end{table}
      

\subsubsection*{Case 2:}

For the first case the following exact solution $u(x,t)$ and $p(x,t)$ was considered

$$u(x,t) = e^{t}(x-x^2)$$
$$p(x,t) = -\sin(\pi x)$$\\

which, for $\nu=1$, gives the PDE below

\begin{equation}
  \label{NS12}
  u_t + u u_x - u_{xx} + p_x = e^t (x-x^2) + e^{2t} (x+x^2) (1-2x) + 2e^t -\pi\cos(\pi x)
\end{equation}

When solving Equation \ref{NS12} with the architectures explained above, the following results were obtained

\begin{table}[H]
  \begin{center}
  \begin{tabular}{ r | c  c | c  c}
    & \multicolumn{2}{c |}{$u(x,t)$} & \multicolumn{2}{c}{$p(x,t)$} \\ \hline
  \textbf{Time} & \textbf{MSE} & \textbf{$l^2$ error}  & \textbf{MSE} & \textbf{$l^2$ error} \\ \hline

  \end{tabular}
  \caption{Results for the first architecture in the second case of the 1-dimensional Navier-Stokes Equations}
  \label{tab:NS121}
  \end{center}
  \end{table}

  \begin{table}[H]
    \begin{center}
    \begin{tabular}{ r | c  c | c  c}
      & \multicolumn{2}{c |}{$u(x,t)$} & \multicolumn{2}{c}{$p(x,t)$} \\ \hline
    \textbf{Time} & \textbf{MSE} & \textbf{$l^2$ error}  & \textbf{MSE} & \textbf{$l^2$ error} \\ \hline
    0.0 & $ 8.6068\times 10^{-7}$ & $ 0.0093 $ & $ 2.8796\times 10^{-3} $ & $ 0.5366$ \\
    0.1 & $ 2.4480\times 10^{-6}$ & $ 0.0156 $ & $ 3.2165\times 10^{-3} $ & $ 0.5671$ \\
    0.2 & $ 3.6892\times 10^{-6}$ & $ 0.0192 $ & $ 3.7059\times 10^{-3} $ & $ 0.6088$ \\
    0.3 & $ 4.4041\times 10^{-6}$ & $ 0.0210 $ & $ 4.3900\times 10^{-3} $ & $ 0.6626$ \\
    0.4 & $ 5.3148\times 10^{-6}$ & $ 0.0231 $ & $ 5.3268\times 10^{-3} $ & $ 0.7298$ \\
    0.5 & $ 6.9956\times 10^{-6}$ & $ 0.0264 $ & $ 6.5958\times 10^{-3} $ & $ 0.8121$ \\
    0.6 & $ 9.9863\times 10^{-6}$ & $ 0.0316 $ & $ 8.3072\times 10^{-3} $ & $ 0.9114$ \\
    0.7 & $ 1.5001\times 10^{-5}$ & $ 0.0387 $ & $ 1.0615\times 10^{-2} $ & $ 1.0303$ \\
    0.8 & $ 2.2718\times 10^{-5}$ & $ 0.0477 $ & $ 1.3734\times 10^{-2} $ & $ 1.1719$ \\
    0.9 & $ 3.3336\times 10^{-5}$ & $ 0.0577 $ & $ 1.7967\times 10^{-2} $ & $ 1.3404$ \\
    1.0 & $ 4.6835\times 10^{-5}$ & $ 0.0684 $ & $ 2.3737\times 10^{-2} $ & $ 1.5407$ \\
    \end{tabular}
    \caption{Results for the second architecture in the second case of the 1-dimensional Navier-Stokes Equations}
    \label{tab:NS122}
    \end{center}
    \end{table}
  
    \begin{table}[H]
      \begin{center}
      \begin{tabular}{ r | c  c | c  c}
        & \multicolumn{2}{c |}{$u(x,t)$} & \multicolumn{2}{c}{$p(x,t)$} \\ \hline
      \textbf{Time} & \textbf{MSE} & \textbf{$l^2$ error}  & \textbf{MSE} & \textbf{$l^2$ error} \\ \hline
      0.0 & $ 1.7844\times 10^{-6}$ & $ 0.0134 $ & $ 5.9847\times 10^{-3} $ & $ 0.7736$ \\
      0.1 & $ 2.4234\times 10^{-5}$ & $ 0.0492 $ & $ 2.0570\times 10^{-2} $ & $ 1.4342$ \\
      0.2 & $ 1.0285\times 10^{-4}$ & $ 0.1014 $ & $ 4.6340\times 10^{-2} $ & $ 2.1527$ \\
      0.3 & $ 2.6243\times 10^{-4}$ & $ 0.1620 $ & $ 8.6240\times 10^{-2} $ & $ 2.9367$ \\
      0.4 & $ 5.3076\times 10^{-4}$ & $ 0.2304 $ & $ 1.4432\times 10^{-1} $ & $ 3.7990$ \\
      0.5 & $ 9.3961\times 10^{-4}$ & $ 0.3065 $ & $ 2.2584\times 10^{-1} $ & $ 4.7522$ \\
      0.6 & $ 1.5272\times 10^{-3}$ & $ 0.3908 $ & $ 3.3733\times 10^{-1} $ & $ 5.8080$ \\
      0.7 & $ 2.3419\times 10^{-3}$ & $ 0.4839 $ & $ 4.8673\times 10^{-1} $ & $ 6.9766$ \\
      0.8 & $ 3.4460\times 10^{-3}$ & $ 0.5870 $ & $ 6.8356\times 10^{-1} $ & $ 8.2678$ \\
      0.9 & $ 4.9188\times 10^{-3}$ & $ 0.7013 $ & $ 9.3924\times 10^{-1} $ & $ 9.6914$ \\
      1.0 & $ 6.8590\times 10^{-3}$ & $ 0.8282 $ & $ 1.2677 $ & $ 11.2590$ \\
      \end{tabular}
      \caption{Results for the third architecture in the second case of the 1-dimensional Navier-Stokes Equations}
      \label{tab:NS13}
      \end{center}
      \end{table}

      \begin{table}[H]
        \begin{center}
        \begin{tabular}{ r | c  c | c  c}
          & \multicolumn{2}{c |}{$u(x,t)$} & \multicolumn{2}{c}{$p(x,t)$} \\ \hline
        \textbf{Time} & \textbf{MSE} & \textbf{$l^2$ error}  & \textbf{MSE} & \textbf{$l^2$ error} \\ \hline
        0.0 & $ 2.4028\times 10^{-6}$ & $ 0.0155 $ & $ 6.6235\times 10^{-3} $ & $ 0.8139$ \\
        0.1 & $ 3.3048\times 10^{-5}$ & $ 0.0575 $ & $ 2.2045\times 10^{-2} $ & $ 1.4848$ \\
        0.2 & $ 1.3703\times 10^{-4}$ & $ 0.1171 $ & $ 5.0095\times 10^{-2} $ & $ 2.2382$ \\
        0.3 & $ 3.4117\times 10^{-4}$ & $ 0.1847 $ & $ 9.4570\times 10^{-2} $ & $ 3.0752$ \\
        0.4 & $ 6.7964\times 10^{-4}$ & $ 0.2607 $ & $ 1.6022\times 10^{-1} $ & $ 4.0027$ \\
        0.5 & $ 1.1959\times 10^{-3}$ & $ 0.3458 $ & $ 2.5301\times 10^{-1} $ & $ 5.0300$ \\
        0.6 & $ 1.9448\times 10^{-3}$ & $ 0.4410 $ & $ 3.8045\times 10^{-1} $ & $ 6.1681$ \\
        0.7 & $ 2.9953\times 10^{-3}$ & $ 0.5473 $ & $ 5.5208\times 10^{-1} $ & $ 7.4302$ \\
        0.8 & $ 4.4333\times 10^{-3}$ & $ 0.6658 $ & $ 7.8000\times 10^{-1} $ & $ 8.8318$ \\
        0.9 & $ 6.3649\times 10^{-3}$ & $ 0.7978 $ & $ 1.0797 $ & $ 10.3908$ \\
        1.0 & $ 8.9200\times 10^{-3}$ & $ 0.9445 $ & $ 1.4709 $ & $ 12.1279$ \\
        \end{tabular}
        \caption{Results for the fourth architecture in the first case of the 1-dimensional Navier-Stokes Equations}
        \label{tab:NS14}
        \end{center}
        \end{table}

\subsection{2-dimentional Navier-Stokes Equations}

For the 2-dimentional Navier-Stokes Equations, only one case was implemeneted with the listed five architectures:

\begin{itemize}
  \item Architechture 1: 5 hidden layers with 50 neurons each
  \item Architechture 2: 8 hidden layers with 20 neurons each
  \item Architechture 3: 8 hidden layers with 50 neurons each
  \item Architechture 3: 8 hidden layers with 70 neurons each
  \item Architechture 5: 8 hidden layers with 100 neurons each
\end{itemize}

\subsubsection*{Case 1:}

For this case, a stream-function forulation was used, were a function $\psi(x,t)$ was defined as $$U = \nabla\times\psi(x,t)$$ therefore

$$u = \psi_y$$
$$v = -\psi_x$$

For this case, the following exact solution $\psi(x,t)$ and $p(x,t)$ was considered

$$\psi(x,t) = (1-x^2y^2)e^{-t}$$
$$p(x,t) = xy$$\\

which, for $\nu=1$, gives the PDE below

\begin{align}
  \label{NS21}
  \begin{split}
    u_t + u u_x + vu_y - u_{xx} - u_{yy} + p_x & = 2x^2ye^{-t} + 12 x^3y^2 e^{-t} + 4ye^{-t} + y \\
    v_t + u v_x + v v_y - v_{xx} - v_{yy} + p_y &= - 2xy^2e^{-t} + 4 x^2y^3 e^{-t} - 4ye^{-t} + x
  \end{split}
\end{align}
 

When solving Equation \ref{NS21} with the architectures explained above, the following results were obtained

\begin{table}[H]
  \begin{center}
  \begin{tabular}{ r | c  c | c  c}
    & \multicolumn{2}{c |}{$u(x,t)$} & \multicolumn{2}{c}{$p(x,t)$} \\ \hline
  \textbf{time} & \textbf{MSE} & \textbf{$l^2$ error}  & \textbf{MSE} & \textbf{$l^2$ error} \\ \hline
  0 & $1.7341\times 10^{-5}$ & $0.8328$ & $2.6630\times 10^{-2}$ & $32.6373$ \\
  0.1 & $1.1645\times 10^{-3}$ & $6.8252$ & $1.0959\times 10^{-2}$ & $20.9369$\\
  0.2 & $3.8980\times 10^{-3}$ & $12.4867$ & $8.0858\times 10^{-3}$ & $17.9842$\\ 
  0.3 & $7.4127\times 10^{-3}$ & $17.2194$ & $1.0201\times 10^{-2}$ & $20.2000$\\ 
  0.4 & $1.1246\times 10^{-2}$ & $21.2102$ & $1.4190\times 10^{-2}$ & $23.8244$\\
  0.5 & $1.5156\times 10^{-2}$ & $24.6217$ & $1.9096\times 10^{-2}$ & $27.6374$\\
  0.6 & $1.9022\times 10^{-2}$ & $27.5841$ & $2.4764\times 10^{-2}$ & $31.4732$\\
  0.7 & $2.2797\times 10^{-2}$ & $30.1976$ & $3.1217\times 10^{-2}$ & $35.3368$\\
  0.8 & $2.6465\times 10^{-2}$ & $32.5360$ & $3.8418\times 10^{-2}$ & $39.2008$\\
  0.9 & $3.0017\times 10^{-2}$ & $34.6508$ & $4.6221\times 10^{-2}$ & $42.9982$\\
  1.0 & $3.3443\times 10^{-2}$ & $36.5749$ & $5.4403\times 10^{-2}$ & $46.6490$\\
  \end{tabular}
  \caption{Results for the first architecture for the 2-dimesnional Navier Stokes Equations}
  \label{tab:NS21}
  \end{center}
  \end{table}
  
  \begin{table}[H]
    \begin{center}
    \begin{tabular}{ r | c  c | c  c}
      & \multicolumn{2}{c |}{$u(x,t)$} & \multicolumn{2}{c}{$p(x,t)$} \\ \hline
    \textbf{time} & \textbf{MSE} & \textbf{$l^2$ error}  & \textbf{MSE} & \textbf{$l^2$ error} \\ \hline
    0.0 & $ 9.9230\times 10^{-5}$ & $ 1.9923 $ & $ 1.5095\times 10^{-2} $ & $ 24.5721$ \\
    0.1 & $ 8.7404\times 10^{-4}$ & $ 5.9128 $ & $ 8.5973\times 10^{-3} $ & $ 18.5443$ \\
    0.2 & $ 2.6156\times 10^{-3}$ & $ 10.2286 $ & $ 7.6538\times 10^{-3} $ & $ 17.4972$ \\
    0.3 & $ 5.0194\times 10^{-3}$ & $ 14.1696 $ & $ 9.5980\times 10^{-3} $ & $ 19.5939$ \\
    0.4 & $ 7.8569\times 10^{-3}$ & $ 17.7278 $ & $ 1.3048\times 10^{-2} $ & $ 22.8455$ \\
    0.5 & $ 1.0956\times 10^{-2}$ & $ 20.9343 $ & $ 1.7481\times 10^{-2} $ & $ 26.4430$ \\
    0.6 & $ 1.4186\times 10^{-2}$ & $ 23.8212 $ & $ 2.2916\times 10^{-2} $ & $ 30.2760$ \\
    0.7 & $ 1.7444\times 10^{-2}$ & $ 26.4155 $ & $ 2.9685\times 10^{-2} $ & $ 34.4588$ \\
    0.8 & $ 2.0647\times 10^{-2}$ & $ 28.7384 $ & $ 3.8272\times 10^{-2} $ & $ 39.1266$ \\
    0.9 & $ 2.3724\times 10^{-2}$ & $ 30.8054 $ & $ 4.9202\times 10^{-2} $ & $ 44.3630$ \\
    1.0 & $ 2.6613\times 10^{-2}$ & $ 32.6271 $ & $ 6.2971\times 10^{-2} $ & $ 50.1880$ \\
    \end{tabular}
    \caption{Results for the second architecture for the 2-dimesnional Navier Stokes Equations}
    \label{tab:NS22}
    \end{center}
    \end{table}

    \begin{table}[H]
      \begin{center}
      \begin{tabular}{ r | c  c | c  c}
        & \multicolumn{2}{c |}{$u(x,t)$} & \multicolumn{2}{c}{$p(x,t)$} \\ \hline
      \textbf{time} & \textbf{MSE} & \textbf{$l^2$ error}  & \textbf{MSE} & \textbf{$l^2$ error} \\ \hline
      0.0 & $ 2.1447\times 10^{-5}$ & $ 0.9262 $ & $ 2.3688\times 10^{-2} $ & $ 30.7818$ \\
      0.1 & $ 1.1856\times 10^{-3}$ & $ 6.8865 $ & $ 1.1292\times 10^{-2} $ & $ 21.2527$ \\
      0.2 & $ 3.8643\times 10^{-3}$ & $ 12.4326 $ & $ 8.9173\times 10^{-3} $ & $ 18.8863$ \\
      0.3 & $ 7.1762\times 10^{-3}$ & $ 16.9425 $ & $ 1.0742\times 10^{-2} $ & $ 20.7285$ \\
      0.4 & $ 1.0635\times 10^{-2}$ & $ 20.6252 $ & $ 1.4303\times 10^{-2} $ & $ 23.9189$ \\
      0.5 & $ 1.4016\times 10^{-2}$ & $ 23.6781 $ & $ 1.8837\times 10^{-2} $ & $ 27.4495$ \\
      0.6 & $ 1.7252\times 10^{-2}$ & $ 26.2692 $ & $ 2.4307\times 10^{-2} $ & $ 31.1812$ \\
      0.7 & $ 2.0354\times 10^{-2}$ & $ 28.5338 $ & $ 3.0877\times 10^{-2} $ & $ 35.1438$ \\
      0.8 & $ 2.3374\times 10^{-2}$ & $ 30.5769 $ & $ 3.8664\times 10^{-2} $ & $ 39.3266$ \\
      0.9 & $ 2.6369\times 10^{-2}$ & $ 32.4774 $ & $ 4.7642\times 10^{-2} $ & $ 43.6540$ \\
      1.0 & $ 2.9401\times 10^{-2}$ & $ 34.2932 $ & $ 5.7633\times 10^{-2} $ & $ 48.0137$ \\
      \end{tabular}
      \caption{Results for the third architecture for the 2-dimesnional Navier Stokes Equations}
      \label{tab:NS23}
      \end{center}
      \end{table}

      \begin{table}[H]
        \begin{center}
        \begin{tabular}{ r | c  c | c  c}
          & \multicolumn{2}{c |}{$u(x,t)$} & \multicolumn{2}{c}{$p(x,t)$} \\ \hline
        \textbf{time} & \textbf{MSE} & \textbf{$l^2$ error}  & \textbf{MSE} & \textbf{$l^2$ error} \\ \hline

        \end{tabular}
        \caption{Results for the fourth architecture for the 2-dimesnional Navier Stokes Equations}
        \label{tab:NS21}
        \end{center}
        \end{table}

      \begin{table}[H]
        \begin{center}
        \begin{tabular}{ r | c  c | c  c}
          & \multicolumn{2}{c |}{$u(x,t)$} & \multicolumn{2}{c}{$p(x,t)$} \\ \hline
        \textbf{time} & \textbf{MSE} & \textbf{$l^2$ error}  & \textbf{MSE} & \textbf{$l^2$ error} \\ \hline
        0.0 & $ 1.6969\times 10^{-5}$ & $ 0.8239 $ & $ 1.9618\times 10^{-2} $ & $ 28.0128$ \\
        0.1 & $ 9.4158\times 10^{-4}$ & $ 6.1370 $ & $ 9.3919\times 10^{-3} $ & $ 19.3824$ \\
        0.2 & $ 3.1448\times 10^{-3}$ & $ 11.2157 $ & $ 8.0040\times 10^{-3} $ & $ 17.8930$ \\
        0.3 & $ 5.9426\times 10^{-3}$ & $ 15.4176 $ & $ 1.0275\times 10^{-2} $ & $ 20.2732$ \\
        0.4 & $ 8.9039\times 10^{-3}$ & $ 18.8721 $ & $ 1.4243\times 10^{-2} $ & $ 23.8686$ \\
        0.5 & $ 1.1787\times 10^{-2}$ & $ 21.7140 $ & $ 1.9205\times 10^{-2} $ & $ 27.7166$ \\
        0.6 & $ 1.4485\times 10^{-2}$ & $ 24.0708 $ & $ 2.4883\times 10^{-2} $ & $ 31.5488$ \\
        0.7 & $ 1.6976\times 10^{-2}$ & $ 26.0586 $ & $ 3.1112\times 10^{-2} $ & $ 35.2773$ \\
        0.8 & $ 1.9294\times 10^{-2}$ & $ 27.7807 $ & $ 3.7755\times 10^{-2} $ & $ 38.8615$ \\
        0.9 & $ 2.1502\times 10^{-2}$ & $ 29.3270 $ & $ 4.4696\times 10^{-2} $ & $ 42.2827$ \\
        1.0 & $ 2.3676\times 10^{-2}$ & $ 30.7742 $ & $ 5.1843\times 10^{-2} $ & $ 45.5382$ \\
        \end{tabular}
        \caption{Results for the fifth architecture for the 2-dimesnional Navier Stokes Equations}
        \label{tab:NS21}
        \end{center}
        \end{table}
  \section{Conclussions}


\section*{Acknowledgements}

\bibliographystyle{vancouver}
\bibliography{references.bib}

\end{document}
